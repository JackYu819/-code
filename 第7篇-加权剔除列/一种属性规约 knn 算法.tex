%\documentclass[11pt,reqno]{article}
\documentclass[11pt]{article}
\usepackage{bbding}
\usepackage{pifont}
\usepackage{amsmath}
\usepackage{mathrsfs}
\usepackage{amsfonts,bm}
\usepackage{amsthm}
\usepackage{epsfig} % 添加 eps 图形
\usepackage[]{harpoon}
\usepackage[active]{srcltx}
\usepackage{indentfirst,latexsym}
\usepackage{psfrag}
\usepackage[]{amssymb}
\usepackage{cite}

\usepackage{pythonhighlight}

% 显示标注信息, 便于分析问题和引用. 成稿时, 对其注释
%\usepackage[notref,notcite]{showkeys}
% 添加包
\usepackage{ctex}
\usepackage{color}
\usepackage{float} % 强制图片位置包

% 添加算法流程框
\usepackage{algorithm}
\usepackage{algorithmic}
\usepackage{multirow}  


\floatname{algorithm}{算法}    % 换成中文
% \renewcommand{\REQUIRE}{输入}
% \floatname{\REQUIRE}{输入} 

\renewcommand{\algorithmicrequire}{\textbf{输入:}}   %改成后面的小标题

\renewcommand{\algorithmicensure}{\textbf{输出:}} 
%\renewcommand{\algorithmiclastcon}{\textbf{Output:}}




%======================================================
\def\pt{\partial}

%=======================================================
\newcommand{\new}{\newcommand*}\new{\rnew}{\renewcommand*}
\new{\newe}{\newenvironment*}\new{\stl}{\setlength}
\stl{\textwidth}{155mm}\stl{\textheight}{22cm}\stl{\headheight}{0cm}
\stl{\topmargin}{0cm}\stl{\oddsidemargin}{0.5cm}\stl{\evensidemargin}{0cm}
\rnew{\arraystretch}{1.2}\rnew{\baselinestretch}{1.2}
\renewcommand{\thefootnote}{\ding{73}}
\newtheorem{thm}{引理}[section]
\newtheorem{cor}{Corollary}
\newtheorem{lem}{Lemma}[section]
\newtheorem{prop}[thm]{Proposition}
\newtheorem{defn}{定义}[section]
\newtheorem{rem}{Remark}
\newtheorem{prob}{Problem}
\newcommand{\norm}[1]{\left\Vert#1\right\Vert}
\newcommand{\abs}[1]{\left\vert#1\right\vert}
\newcommand{\set}[1]{\left\{#1\right\}}
\newcommand{\Real}{\mathbb R}
\newcommand{\eps}{\varepsilon}
%\section{\ewcommand{\To}{\longrightarrow}}
%\begin{center}
%    
%\end{center}

\newcommand{\BX}{\mathbf{B}(X)}
\newcommand{\A}{\mathcal{A}}
\newcommand{\ts}{\textstyle}
\newcommand{\tg}{\mbox{\rm tg}}
\newcommand{\ctg}{\mbox{\rm ctg}}
\newcommand{\atg}{\tg^{-1}}
\newcommand{\actg}{\ctg^{-1}}
\newcommand{\asin}{\sin^{-1}}
\newcommand{\acos}{\cos^{-1}}
\newcommand{\dps}{\displaystyle}
\newcommand{\fnz}{\footnotesize}
\newcommand{\D}{\displaystyle}
\newcommand{\fr}{\frac}
\newcommand{\scp}{\scriptstyle}
\newcommand{\pa}{\partial}
\newcommand{\vphi}{\varphi}
%%%%%%%%%%%%%%%%%%%%%%%
\renewcommand{\theequation}{\thesection.\arabic{equation}}
\numberwithin{equation}{section}
%%%%%%%%%%%%%%%%%%%%%%%


%% Title  ===========================================================
\newe{keywords}
   {\begin{quote}{\bf 关键词:}}
      {\end{quote}}
%\newe{AMS}
%%   {\begin{quote}{\bf AMS subject classification 2000:}}
%%      {\end{quote}}
%\newe{MSC}{\vspace*{5mm}
%    {\noindent\it Mathematics Subject Classification(2000):}}{}
%\new{\sect}[1]{\section{#1}\setcounter{equation}{0}
% \setcounter{thm}{0}\setcounter{lmm}{0}\setcounter{rmk}{0} }

\newe{abstract1}
   {\begin{quote}{\bf 摘 \ 要:}}
      {\end{quote}}


%%% 英语
\newe{keys}
{\begin{quote}{\bf Keywords: }}
{\end{quote}}

\newe{abst}
{\begin{quote}{\bf Abstract: }}
{\end{quote}}





\begin{document}
%% ==================== Title =======================================

\title{一种有效的VSKNN算法}

\author{
于祥雨$^{1,\dag}$ %, Guodong Wang$^{c}$
\\{\small \it $^1$ 杭州师范大学理学院 , 杭州 \quad 310036}
}
%\\
%{\small \it $^b$School of Mathematical Sciences, Fudan University, Shanghai, 200433, PR China}
%\\
%{\small \it $^c$School of Mathematics \& Physics, Anhui Jianzhu University, Hefei, 230601,
%PR China}}

\rnew{\thefootnote}{\fnsymbol{footnote}}

\footnotetext{ $\dag$作者简介: 于祥雨(1989-), 男, 山东郓城人, 硕士研究生, 主要研究方向为大数据分析.}
\footnotetext{ E-mail : xiangyuyu819@yeah.net }


\date{}


\maketitle
%%========================= abstract ==================================
\begin{abstract1}
维规约有属性(特征)选择和维变换两个方面, 排除无关属性可以有效提高分类准确率. 本文主要基于属性选择提出一种有效的 VSKNN 算法, 
该算法根据给定阈值确定无关属性, 再通过保留特征的数据进行 KNN 算法. 本文通过对 KNN 算法和 VSKNN 算法在 7 个数据集进行实验, 结果表明 VSKNN 算法是有效的.
\end{abstract1}

\begin{keywords}
KNN; 分类; 维规约
\end{keywords}


\begin{abst}
Dimensionality conventions have two properties of feature selection and dimension transformation, excluding irrelevant attributes can effectively improve the classification accuracy. This paper mainly proposes an effective VSKNN algorithm based on attribute selection, which determine irrelevant attributes according to a given threshold, and then the KNN algorithm is carried out by preserving the attribute data. Through experiments on KNN algorithm and VSKNN algorithm in seven data sets, the results show that the VSKNN algorithm is effective.
\end{abst}

\begin{keys}
KNN; Classification; Victoria Statute
\end{keys}

%\begin{AMS}
%35L20, 35L70, 35L80.
%\end{AMS}
 
%%$$$$$$$$$$$$$$$$$$$$$$$$$$$$$$$$ section 1 %$$$$$$$$$$$$$$$$$$$$$$$$$$$$$$$
\section{引言} \label{引言}
随着大数据概念的兴起, 数据挖掘\cite{毛国君}, 机器学习\cite{陈康}和人工智能\cite{孙伟平}研究领域备受关注.
最近邻规则分类 (K-Nearest Neighbor, KNN) 算法\cite{knn}作为数据挖掘十大经典算法之一, 
在实际问题中具有广泛的应用.
在数据挖掘方面, 虽然大数据集具备更好的挖掘潜力, 但是依然无法保证挖掘效果比小型数据集的效果好.
数据集特征数量并非越多越好, 有的特征对结果的影响很小, 还有可能存在干扰结果的特征以及特征之间存在
高度的相关性, 因此合理, 适当, 科学的维规约\cite{许明旺}对分类结果有显著的影响. 

目前,关于维规约的方法有多种. 比如: 通过相关性来剔除不相关或高度相关的特征, 主成分分析法(Principal Component Analysis, PCA)\cite{pca}以及局部线性嵌入(Locally Linear Embedding, LLE)\cite{saul}等.
不同维规约方法都有不同的适应情况, 比如: 通过相关性可以降低维数使得回归或分类效果更好, 主成分分析法主要是一种数据预处理方法,
局部线性嵌入是一种非线性降维方法, 对于具有拓扑结构的数据具有较好的效果.
本文基于同类之间数据特征高度相似, 不同类之间数据特征差异较大的思想, 提出一种同类别数据中同维度的数据趋于一致的迭代算法, 进而获得各个类对应的权向量, 将各个类的权向量构造成一个矩阵, 通过统计学原理计算其均值和标准差来确定需要剔除的特征, 对于剔除后
数据集进行 KNN 算法, 即维规约 KNN 算法(Victoria Statute KNN, VSKNN), 实验结果表明本文提出的算法在部分数据集上比传统 KNN 算法具有更好的效果.



\section{预备知识}\label{预备知识}
% 本小节主要阐述本文研究期间采用的思想和方法, KNN 算法, 迭代格式.

\subsection{留出法}
留出法(hold-out)\cite{机器学习原理}是将数据集 $D$ 划分为两个互斥的集合, 分别记为训练集 $S$, 测试集 $T$, 即: $D = S\bigcup T$, $S\bigcap T = \emptyset$ .

留出法在操作过程中, 需要尽量保证划分的训练集和测试集的分布一致, 这样才能避免因划分过程引入额外的偏差进而对结果产生影响.

单次采用留出法得到的估计结果不稳定\cite{机器学习原理}, 因此需要通过随机抽样的方法进行多次留出法, 以估计结果的平均值作为该算法的准确率. 另外, 训练集和测试集划分比例不同对评估结果有不同的影响. 训练集数量过大, 其预测结果往往不够稳定, 准确. 若训练集数量过少, 训练集 $S$ 与数据集 $D$ 之间会造成很大差别, 会严重降低预测结果的保真性(Fidelity), 针对这一问题尚未有好的解决方案.


\subsection{KNN 算法}
已知类别数据集(训练集)$T_{train} = \{(x_{i}, y_{i})\}_{i=1}^{n}$, 对未知类别数据集$T_{test}$进行分类. 下面给出KNN算法\cite{knn}的伪代码:

\begin{algorithm}[H]   
\caption{\quad KNN 算法}   
% \label{alg:算法框架}   
\begin{algorithmic}[1] %这个1 表示每一行都显示数字  
\REQUIRE  ~~\\ %算法的输入参数:Input 
训练集 $T_{train}$ 和 $k$ 值,\\  
未知类别数据集 $T_{test}$ .

\ENSURE  ~~\\ %算法的输出:Output
%未知类别数据集 $T_{test}$ 以及对应的类别标签.
未知类别数据集$T_{test}$对应的分类标签
% \STATE 对未知类别数据集中的第 $i$ 样本$t^{i}_{test}$执行以下操作:
\FOR{$i$}
\STATE 计算未知类别样本 $t^{i}_{test}$ $\in T_{test}$与$T_{train}$ 的欧式距离(或其他距离);   
% \label{ code:fram:extract }%对此行的标记,方便在文中引用算法的某个步骤  
\STATE 将距离以升序的方式进行排序;   
% \label{code:fram:trainbase}  
\STATE 选取前 $k$ 个已知类别数据集的数据;   
% \label{code:fram:add}  
\STATE 统计前 $k$ 个数据所属不同类别出现的频数;   
% \label{code:fram:classify}  
\STATE 频数最大的类别的标签作为未知类别样本 $t^{i}_{test}$ 的标签.
\ENDFOR
% \RETURN $E_n$; %算法的返回值
\RETURN \\
$T_{test}$ 数据集对应的类别标签
\end{algorithmic}  
\end{algorithm}  


\subsection{迭代式}
\begin{defn}
给定一初始权重$W_{0} = (w_{1}, w_{2}, ..., w_{m})$, $w_{i}>0 $, $\sum_{i=1}^{m}w_{i} = 1$, 对于第$j$类训练集子集$T^{j}$
\begin{equation}
\left\{\begin{array}{c}
T^{j}_{n+1}= T^{j}_{n}W_{n}  \\
%W_{0} = \alpha \times argmax(W_{n-1}) 
W_{n} = \frac{\alpha \cdot arg ~ max(W_{n-1})}{||\alpha \cdot arg ~ max(W_{n-1})||}
% \alpha \cdot arg ~ max(w_{1}, w_{2}, ..., w_{m})||}
\end{array}\right.
\end{equation}
其中, $\alpha \in(0,1)$是一个常数, 也可以是一个关于迭代步数的递减函数, $\alpha$ 的取值大小影响收敛速度和效果;$||\cdot||$是二范数, 即每次迭代均要将 $W$ 进行归一化.
\end{defn}

针对以上迭代式, 需要给出合适的收敛条件, 其收敛条件
\begin{equation}
     ||max(std(T^{j}_{n+1})) - min(std(T^{j}_{n+1}))|| < \epsilon
 \end{equation} 
其中, $std(T^{j}_{n+1})$为第$j$类训练集迭代$n+1$次后矩阵同维度数据的标准差的$1\times m$向量, $max(std(T^{j}_{n+1}))$和$min(std(T^{j}_{n+1}))$分别是向量中的最大值和最小值; $\epsilon$ 为允许误差, 本文取值为$1\times 10^{-4}$.

\section{VSKNN 算法}
\subsection{算法原理}
根据定义 2.1 计算训练集中各个类对应的最佳权重, 然后根据最佳权重矩阵, 计算各个特征对应的算术平均数, 标准差. 本文通过均值累计阈值预先筛选出剔除
的特征, 再结合各个维度的标准差, 综合筛选出需要剔除的特征.
本文以 Iris 数据集为例($\epsilon = 1
\times 10^{-4}$, $\alpha = 0.02$, $max\_iter = 10000$), 其结果如表所示.

\begin{table}[H]
  \centering 
    \caption{Iris 数据集不同类对应的权重}
\begin{tabular}{lllll}
\hline
Label &       SepalLength &   SepalWidth &  PetalLength &  PetalWidth \\
\hline
setosa     &    0.331126 &    0.00662252 &    0.331126 &    0.331126 \\
versicolor &    0.00980392 &    0.490196 &    0.00980392 &    0.490196 \\
virginica  &    0.0188679 &    0.0188679 &    0.0188679 &    0.943396 \\
\hline
\end{tabular}
\end{table}

通过表1显示: 对于类别 setosa 而言, SepalLength, PetalLength 和 PetalWidth 贡献值最大, 均为 0.331126, SepalWidth 贡献值最小; 
对于类别 versicolor 而言, SepalWidth 和 PetalWidth 贡献值最大, 其次是 SepalLength 和 PetalLength 一致; 对于类别 virginica 而言, 
PetalWidth 贡献值最大, 其次是 SepalLength, SepalWidth 和 PetalLength.

\begin{table}[H]
  \centering 
    \caption{Iris 基本统计量}
\begin{tabular}{llll}
\hline
特征 &      均值 &        累计值(均值) &       标准差 \\
\hline
PetalWidth &  0.588239 &  0.588239 &  0.317692 \\
SepalWidth &  0.171896 &  0.760135 &  0.275724 \\
PetalLength &  0.119933 &  0.880067 &  0.182955 \\
SepalLength &  0.119933 &  1.000000 &  0.182955 \\
\hline
\end{tabular}
\end{table}

通过表2显示: 特征 PetalWidth 的值最大, 高达 0.588239, 其标准差为 0.317692, 也是所有特征中最大的; 特征 SepalWidth 的值次之, 均值为 0.171896, 标准差为 0.275724; 特征 PetalLength 和 SepalLength 一致, 均值都为 0.119933, 标准差均为 0.182955.

根据给定的累计值阈值 $\sigma = 0.98$, (a)由于 PetalLength 的累计值为 0.880067, SepalLength 的累计值为 1.000000, 因此初步剔除 1 个特征 SepalLength; (b)根据初步剔除特征个数 $n = 1$, 筛选出前 $n$ 个最小的标准差, 即为 SepalLength. 通过方法 (a) 和 (b) 求交集, 作为需要剔除的特征, 然后再对剔除特征的数据集进行 KNN 算法.


\subsection{伪代码}
本小节给出本文算法的伪代码, 如下所示.
\begin{algorithm}[H]   
\caption{\quad VSKNN 算法}   
% \label{alg:算法框架}   
\begin{algorithmic}[1] %这个1 表示每一行都显示数字  
\REQUIRE  ~~ \\ %算法的输入参数:Input 
训练集 $T_{train}$, 参数 $\alpha$ ,最大迭代步数 $iter_{max}$ \\
误差$\epsilon$, 未知类别数据集 $T_{test}$, 初始权重$W_{0}$, 累计阈值$\sigma$.

\ENSURE  ~~\\ %算法的输出:Output
%未知类别数据集 $T_{test}$ 以及对应的类别标签.
未知类别数据集$T_{test}$对应的分类标签
% \STATE 对未知类别数据集中的第 $i$ 样本$t^{i}_{test}$执行以下操作:
% 添加改进算法, 中心距
% \FOR{$l \in C$}
% \STATE 计算第 $l$ 类数据的中心向量$v_{l}$;
% \ENDFOR
% \STATE 存储中心向量训练集 $V = \{(v_{l}, y_{l})\}^{c}_{l=1}$.
\FOR{$j \in C$}
\IF{$||max(std(T^{j})) - min(std(T^{j}))|| < \epsilon$}
\STATE $\epsilon$ 值过大, 结束遍历, 调整参数$\epsilon$, 重新计算.
\ELSE 
\WHILE{$iter < iter_{max}$ and $||max(std(T^{j}_{n+1})) - min(std(T^{j}_{n+1}))|| > \epsilon$}
\STATE 通过定义(2.1)求解第 $j$ 类数据的迭代最佳随机向量$w_{j}$以及$b_{j}$.
\ENDWHILE
\ENDIF
\ENDFOR
\STATE 存储迭代后不同类别的 $new\_W = (b,W) = \{(b_{j},w_{j})\}_{j=1}^{c}$, 如表1格式.
\STATE 计算 $new\_W$ 的算术平均数, 累计值和标准差, 并以累计值排序(升序).
\STATE 根据阈值$\sigma$, 方法 (a) 和(b) 确定删除特征.
\STATE 对训练集 $T_{train}$ 和未知类别数据集 $T_{test}$ 删除确定的特征, 获得新的训练集 $T^{'}_{train}$ 和新未知类别数据集 $T^{'}_{test}$.
\STATE 通过算法1(KNN算法)对$T^{'}_{test}$ 在训练集 $T^{'}_{train}$ 上进行分类.
% \RETURN $E_n$; %算法的返回值
\RETURN \\
$T_{test}$ 数据集对应的类别标签
\end{algorithmic}  
\end{algorithm}


\section{实验结果与分析}
\subsection{实验数据集}
本文从 UCI 机器学习知识库(http://archive.ics.uci.edu/ml/index.php)获取7个数据集, 分别为:  Iris, Balance, Wine , Glass,  Segmentation,  Winequality-white 以及 Winequality-red. 关于数据集的整体信息描述详见表3. 为确保实验结果的合理性与科学性, 本文实验选择的训练集和测试集均为随机抽选, 将每个数据集样本量的 80\%作为训练集, 20\%作为测试集.

%另外, 针对
% breast-cancer-wisconsin 数据集总共有699个样本, 由于部分样本存在缺失值, 本文实验中将含有缺失值的样本进行剔除, 整理为683个样本数据集.
%总共有699个样本, 由于16样本含有缺失值, 本文实验中将缺失值数据进行移除, 仅实验研究683个样本.

本文实验通过 Python3 软件编程实现, 运行系统:OS X EI Capitan, 处理器: 1.6GHz intel Core i5, 内存: 8GB 1600MHz DDR3.

\begin{table}[H]
  \centering 
    \caption{UCI机器学习知识库数据集和相关说明}\label{表1说明}
  \begin{tabular}{lrrrrr}
\hline
% after \\ : \hline or \cline{col1-col2} \cline{col3-col4} ...
数据集 & 样本数量 & 特征 & 类别数 & 训练集样本量 & 测试集样本量\\
 \hline
Wine & 178 & 13 & 3 & 142 & 36 \\
Iris & 150 & 4 & 3 & 120 & 30 \\
Glass & 214 & 9 & 6 & 150 & 64\\
Balance & 625 & 4 & 3 & 500 & 125\\
Segmentation & 210 & 19 & 7 & 168 & 42 \\
Winequality-white & 4898 & 11 & 7 & 3918 & 980 \\
Winequality-red & 1599 & 11 & 6 & 1279 & 320 \\
\hline
\end{tabular}
\end{table}

\subsection{实验结果}
本小节通过留出法获取训练集和测试集, 再通过传统 KNN 算法和 VSKNN 算法进行对比. K 值取值范围为1-20, 实验给出分类准确率
最高的结果及对应的 K 值. 各个数据集的实验结果如下表所示.

\begin{table}[H]
  \centering 
    \caption{各个数据集计算结果}\label{结果}
  \begin{tabular}{llllr}
\hline
% after \\ : \hline or \cline{col1-col2} \cline{col3-col4} ...
数据集 & KNN(K) & VSKNN(K) & del\_num & $\sigma$ \\
 \hline
Iris & 100.00(5) & 98.11(9) & 1 & 0.98000 \\
Wine & 84.13(1) & 92.06(1) & 1 & 0.99500 \\
Glass & 66.67(1) & 66.67(1) & 0 & 0.98000 \\
Balance & 90.87(18) & 79.91(12) & 1 & 0.95000 \\
Segmentation & 79.73(1) & 89.19(12) & 2 & 0.99999 \\
Winequality-white & 55.51(1) & 57.38(1) & 1 & 0.99990 \\
Winequality-red & 53.57(1) & 59.64(14) & 3 & 0.99950 \\
\hline
均值 & 75.76 & 77.57 & $-$ & $-$ \\
\hline
\end{tabular}
\end{table}
通过表4可以看出, Iris 数据集中, 当 K = 5时, KNN 算法的分类准确率为100\%,
高于 VSKNN 算法的98.11\%, VSKNN 算法剔除了1个特征, 其3个特征的 KNN 算法没有原始结果好;
Wine 数据集中, 当 K = 1 时, KNN 算法的分类准确率为 84.13\%, 低于 VSKNN 算法的92.06\%,
VSKNN 算法剔除1个特征, 但是分类准确率有明显的提高; Glass 数据集中, KNN 算法在 K = 1 时
分类准确率最高, VSKNN 算法在参数 $\sigma = 0.98000$ 下判断没有可以剔除的特征;
Balance 数据集中, KNN 算法在 K = 18 时分类准确率最高, 为90.87\%, 远高于 VSKNN 算法在 K = 12 时
的79.91\%; Segmentation 数据集中, KNN 算法在 K = 1 时分类准确率最高, 为79.73\%, 远低于 VSKNN 算法(K=12)
的89.12\%, 并且剔除2个特征; Winequality-white 和 Winequality-red 两个数据中, VSKNN 算法略好于传统 KNN 算法, 
Winequality-red 数据集剔除了3个特征, Winequality-white 剔除了1个特征.


\subsection{实验分析}
本文通过对7个数据集进行实验, KNN 算法的平均分类准确率为 75.76\%, VSKNN 算法的平均分类准确率为 77.57\%, 整体而言, VSKNN 算法的效果并没有显著提高, 但是对 Wine 和 Segmentation 数据集有显著的提高.

% 针对 Iris 数据集, 通过定义 2.1 计算后的权重矩阵中, PetalWidth 特征的均值相比于其他3个特征较大, 标准差也最大, 而 
% PetalLength 和 SepalLength 特征的均值最小, 标准差也最小, 和标准差, 各个类别对应的标准差, 

本文算法相比于 KNN 算法在计算量上较大, 主要体现在无关特征的筛选上. 另外, 迭代算法的效率受参数 $\alpha$ 和误差 $\epsilon$
的影响, 若 $\alpha$ 设定过小, 会导致收敛速度较低, 若$\alpha$ 设定过大, 可能在给定的误差范围内无法收敛, 因此需要根据实际情况进行调整. 再者, 累计阈值 $\sigma$ 的大小对 VSKNN 算法分类准确率有直接的影响, $\sigma$ 设定较小会致使
无关特征数量较多, 使得保留特征数量较小或少于分类类别数量, 进而致使分类效果较少, $\sigma$ 设定较大导致无关特征数量较少, 或无剔除特征, 最终导致算法无效. 


% \section{结束语}\label{结束语}



\begin{thebibliography}{99}
\addtolength{\itemsep}{-0.7em}

\bibitem{knn} Weinberger K Q, Saul L K. Distance Metric Learning for Large Margin Nearest Neighbor Classification[J]. Journal of Machine Learning Research, 2009: 207-244.

\bibitem{pca}  Wold S, Esbensen K, Geladi P. Principal component analysis[J]. Chemometrics and intelligent laboratory systems, 1987, 2(1-3): 37-52.

\bibitem{saul} Roweis S T, Saul L K. Nonlinear dimensionality reduction by locally linear embedding[J]. science, 2000, 290(5500): 2323-2326.

\bibitem{毛国君} 毛国君. 数据挖掘原理与算法[M]. 北京: 清华大学出版社, 2005.

\bibitem{许明旺} 许明旺, 施润身. 维规约技术综述[J]. 计算机应用, 2006, 26(10): 2401-2404.

\bibitem{孙伟平} 孙伟平. 关于人工智能的价值反思[J]. 哲学研究, 2017 (10): 120-126.

\bibitem{陈康} 陈康, 向勇, 喻超. 大数据时代机器学习的新趋势[J]. 电信科学, 2017, 28(12): 77-85.

\bibitem{机器学习原理} 周志华. 机器学习[M]. 北京: 清华大学出版社, 2016.

\end{thebibliography}
\end{document}
